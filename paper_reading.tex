\documentclass{article}
\usepackage{graphicx}
\usepackage{float}
\begin{document}
\title{Paper Reading}
\author{Eazine Huang}
\date{\today}
\maketitle

\part{Real-time fluorescence and deformability cytometry}

The throughput of cell mechanical characterization has recently approached that of conventional flow cytometers. However, this very sensitive, label-free approach still lacks the specificity of molecular markers. Here we developed an approach that combines real-time 1D-imaging fluorescence and deformability cytometry in one instrument (RT-FDC), thus opening many new research avenues. We demonstrated its utility by using subcellular fluorescence localization to identify mitotic cells and test for mechanical changes in those cells in an RNA interference screen.

%\begin{figure}[H]
%\centering
%\includegraphics[width=3.54in,height=4in]{1.jpg}
%\end{figure}

1.To test whether cell mechanic can be used as a phenotypic marker for human CD34+ HSPCs. In the narrow constriction zone, cells deform as a result of hydrodynamic interaction. On the basis of the fluorescence intensity, classify the cells into CD34+ and CD34-. The deformation versus area is recorded to copmare CD34+ cells and CD34- cells.
Deformation=1-2sqrt(pi*Area)/Perimeter

2.The plot of determition versus area for CD34- cells shows more spread than that for CD34+.
It is a way to sort the cells based on the determition.


3.HeLa cells could be separated into metaphase and anaphase populations on the basis of the presence of single and double peaks of mCherry fluorescence signal.

\emph{Innovations}: combine the sensitivity of the measured cell mechanics with molecular specificity of fluorescent probes.

\part{Cell-free extract based optimization of biomolecular circuits with droplet microfluidics}

Engineering an efficient biomolecular circuit often requires time-consuming iterations of optimization. Cell-free protein expression systems allow rapid testing of biocircuits in vitro, speeding the design–build–test cycle of synthetic biology. In this paper, we combine this with droplet microfluidics to densely scan a transcription–translation biocircuit space. Our system assays millions of parameter combinations per hour, providing a detailed map of function. The ability to comprehensively map biocircuit parameter spaces allows accurate modeling to predict circuit function and identify optimal circuits and conditions. 

%\begin{figure}[H]
%\centering
%\includegraphics[width=6in,height=4in]{2.jpg}
%\end{figure}

1.What is biocircuit? 

Different biomodules interact with each other to decide protein expression.
In this paper, circuit is IFFL(incoherent feedforward loop), refered to paper 12. It is the biological theorey basis for this paper. 


2.This paper combines cell-free protein expression with droplet microfluidics for ultrahigh-throughput scanning of transcription–translation based biomolecular circuits.


3.Use pressure controller system  to design different period sinusoidal functions in order to independetly modulate each inlet and change the composition of droplets. The droplet contents and volumes are observed by  fluorescent detection in three channels.
The key point is that droplet volume should remain constant. 
Each droplet volume is detected by time duration.


\part{Optofluidic marine phosphate detection with enhanced absorption using a Fabry–Pérot resonator}

Real-time detection of phosphate has significant meaning in marine environmental monitoring and fore- casting the occurrence of harmful algal blooms. Conventional monitoring instruments are dependent on artificial sampling and laboratory analysis. They have various shortcomings for real-time applications be- cause of the large equipment size and high production cost, with low target selectivity and the requirement of time-consuming procedures to obtain the detection results. We propose an optofluidic miniaturized analysis chip combined with micro-resonators to achieve real-time phosphate detection. The quantitative water-soluble components are controlled by the flow rate of the phosphate solution, chromogenic agent A (ascorbic acid solution) and chromogenic agent B (12\% ammonium molybdate solution, 80\% concentrated sulfuric acid and 8\% antimony potassium tartrate solution with a volume ratio of 80: 18: 2). Subse- quently, an on-chip Fabry–Pérot microcavity is formed with a pair of aligned coated fiber facets. With the help of optical feedback, the absorption of phosphate can be enhanced, which can avoid the disadvan- tages of the macroscale absorption cells in traditional instruments. It can also overcome the difficulties of traditional instruments in terms of size, parallel processing of numerous samples and real-time monitoring, etc. The absorption cell length is shortened to 300 μm with a detection limit of 0.1 μmol L−1. The time re- quired for detection is shortened from20min to 6 seconds. Predictably, microsensors based on optofluidic technology will have potential in the field of marine environmental monitoring.

%\begin{figure}[H]
%\centering
%\includegraphics[width=3.54in,height=4in]{3.jpg}
%\end{figure}

1.How to form FP cavity?

The facts of the fibers are coated with 40 and 60 nm Au using an electron beam evaporator. The reflectic indexs of the two facts are 65\% and 74\%.


2.How to achieve strict alignment?

There is an iron baseplate and a magnetic briquette. In the iron baseplate, there is a groove that matches the fiber. First, the plated fibers are placed in the groove of the fiber aligner. Then, we move the baseplate carefully to form the fibers and the reserved channels are aligned under a microscope.


3.Use microfluidic chip to mix phosphate solution and chromogenic agent. The theory is Beer-Lamber law. The method in this paper costs less time and has higher detection range which can be used to monitor marine environment.


4.improvement:change the distance of the cavity or use other resonators such as ring resonators, distributed feedback resonators


\part{A sharp-edge-based acoustofluidic chemical signal generator}

Resolving the temporal dynamics of cell signaling pathways is essential for regulating numerous downstream functions, from gene expression to cellular responses. Mapping these signaling pathways requires the exposure of cells to time-varying chemical signals; these are difficult to generate and control over a wide temporal range. Herein, we present an acoustofluidic chemical signal generator based on a sharp-edge-based micromixing strategy. The device, simply by modulating the driving signals of an acoustic transducer including the ON/OFF switching frequency, actuation time and duty cycle, is capable of gener- ating both single-pulse and periodic chemical signals that are temporally controllable in terms of stimulation period, stimulation duration and duty cycle.We also demonstrate the device's applicability and versatility for cell signaling studies by probing the calcium (Ca2+) release dynamics of three different types of cells stimulated by ionomycin signals of different shapes. Upon short single-pulse ionomycin stimulation (∼100 ms) generated by our device, we discover that cells tend to dynamically adjust the intracellular level of Ca2+ through constantly releasing and accepting Ca2+ to the cytoplasm and from the extracellular environment, respectively. With advantages such as simple fabrication and operation, compact device design, and reliability and versatility, our device will enable decoding of the temporal characteristics of signaling dy- namics for various physiological processes.

%\begin{figure}[H]
%\centering
%\includegraphics[width=3.54in,height=4in]{4.jpg}
%\end{figure}

1.Generate single pulse and periodic chemical signals with different period, duration and cycle  based on a sharp-edge-based micromixing strategy by modulating the driving signals of an acoustic transducer.


Mix buffer and stimulant solution(FITC agent) with the generator. The fluorecent intense represents the signal generated by the transducer based on the sharp-edge structure. It is the rule to modulate the acoustic transducer.The driving frequency of 4.5 kHz and the driving voltage of 20 Vpp were used in all experiments.


2.With this signal generator,study on the calcuim release dynamics of three different types of cells.


3.The fluid is not continuous? 

No! In the video, all the cells are in fixed position. The medium and stimulant are injected to the channel after the cells adhere to the bottom of the channel.  

\part{Real-time detection and monitoring of the drug resistance of single myeloid leukemia cells by diffused total internal reflection}


Real-time detection and monitoring of the drug resistance of single cells have important significance in clinical diagnosis and therapy. Traditional methods operate a number of times for each individual concen- tration, and innovation is required for the design of more simple and efficient manipulation platforms with necessary higher sensitivity. Here, we have developed a novel diffused total internal reflection (TIR) method to perform drug metabolism and cytotoxicity analysis of trapped myeloid leukemia cells. Molm-13 cells, a type of acute myeloid leukemia cell, were chosen and injected into the device and fittingly captured by cell traps. Differing from previous studies, a series of different concentrations of azelaic acid (AZA) drug could be used from 0 mM to 50 mM through convection and diffusion processes in a single chip, with each con- centration region featuring 50 cells, with a total of 549 cell trapping units. Thanks to the high sensitivity of the TIR method, only cells with the same drug concentration could be illuminated in the detection process. By adjusting the incident angle, we could exactly detect and monitor the drug resistance of the cells using different drug concentrations and the experimental resolution of the drug concentration was as small as 5 mM. Images of the membrane integrity and morphology of the cells in the bright field were measured and we also monitored the cell viabilities in the dark field over 2 hours. The effects of AZA on the Molm-13 cells were explored in different concentrations at the single cell level. Compared with the results of the tradi- tional MTT assay method, the experimental results are more simple and accurate. A cell death of 5\% at an AZA concentration of 5 mM was observed after 30 minutes, while a concentration of 40 mM corresponded to a 98\% cell death. The designed method in this study provides a novel toolkit to control and monitor drug resistance at the single cell level more easily with higher sensitivity and we believe it has significant potential application in single cell quality assessment and medicine analysis in clinical practice.

%\begin{figure}[H]
%\centering
%\includegraphics[width=3.54in,height=4in]{5.jpg}
%\end{figure}


1.This microfluidic device is used to detect and monitor AZA drug resistance  of single Molm-13 cell with total internal reflection(TIR). Different concentrations of the AZA drug have a different refractive index and can be pumped at the same time. The cells trapped by the trapping structure can be detected by adjust the incident angle.

2.Two layer photoresist. The height of the trapping struceture is lower than that of main channel.

3.MTT is used to study the effect of AZA concentrations and time on cell viabilities compared to the experimental data. 


\part{Multiparameter cell-tracking intrinsic cytometry for single-cell characterization}

An abundance of label-free microfluidic techniques for measuring cell intrinsic markers exists, yet these techniques are seldom combined because of integration complexity such as restricted physical space and incompatible modes of operation. We introduce a multiparameter intrinsic cytometry approach for the characterization of single cells that combines ≥2 label-free measurement techniques onto the same platform and uses cell tracking to associate the measured properties to cells. Our proof-of-concept implementation can measure up to five intrinsic properties including size, deformability, and polarizability at three frequencies. Each measurement module along with the integrated platform were validated and eval- uated in the context of chemically induced changes in the actin cytoskeleton of cells. viSNE and machine learning classification were used to determine the orthogonality between and the contribution of the mea- sured intrinsic markers for cell classification.
%\begin{figure}[H]
%\centering
%\includegraphics[width=3.54in,height=4in]{6.jpg}
%\end{figure}

1.deformability module design and analysis

This measurement platform was introduced by Fletcher Group. Cells flow from the wide channel to the narrow channel. Different cells have different deformability. The key point is to detect the transmit time that the cells get through the region of interest(ROI). The higher deformability, the less transmit time. Use solutions to deal with cells to increase or decrease the defomability of cells. 

2.size and polarizability module design and analysis

The cells appear at a balance position with the fluid drag force and the negative DEP force. Cell polarizability is reported as the real part of the Re[CM]. It is relevant to the frequency of the electric field.

3.According to visual interactive stochastic neighbor embedding(viSNE), there is no correlation to each other. Futhermore, other algorithms are used to analyze the experimental data. It is similar to viSNE. There is no doubt that I can not understand.

4.The main advantage of this cytometry is that the platform can detect 5 parameters.
The  field-of-review of the microscope with 5X objective is not entirely used.  

\part{A linear concentration gradient generator based on multi-layered centrifugal microfluidics and its application in antimicrobial susceptibility testing}

In almost any branch of chemistry or life sciences, it is often necessary to study the interaction between different components in a system by varying their respective concentrations in a systematic manner. Cur- rently, many procedures for generating a series of samples of different solute concentration levels are still done manually by dilution. To address this issue, we present herein a highly automated linear concentration gradient generator based on centrifugal microfluidics. The operation of this device is based on the use of multi-layered microfluidics in which individual fluidic samples to be mixed together are stored and metered in their respective layers before finally being transferred to a mixing chamber. To demonstrate the opera- tion of this scheme, we have used the device to conduct antimicrobial susceptibility testing (AST). Firstly, DI water, ampicillin solution and E. coli suspension were loaded into the chambers in different layers. As the device went through several rounds of spinning at different speeds, a series ofmetered dosages of ampicil- lin along a linear concentration gradient were introduced to the mixing chamber and mixed with E. coli au- tomatically. By monitoring the spectral absorbance of the suspensions, we were able to establish the mini- mum inhibitory concentration (MIC) value of ampicillin against E. coli. The process took about 3 hours to complete, and the experimental results showed a strong correlation with those obtained with the standard CLSI broth dilutionmethod. Clearly, the platform is useful for a wide range of applications such as drug discovery and personalised medicine, where concentration gradients are of concern.

%\begin{figure}[H]
%\centering
%\includegraphics[width=5in,height=4in]{7.jpg}
%\end{figure}

1.How to build air vent?

2.How to bond the PDMS layers?

3.Three layers microfluidic device is used to generate linear gradient concentration. There are three inlets injecting different solutions to metering chambers with low rpm. With high rpm, solutions can break the air vents, inject to the volumes and mix with other solutions to change concentrations.

\part{surfactans in droplets-based microfluidics}

Surfactants are an essential part of the droplet-based microfluidic technology. They are involved in the stabilization of droplet interfaces, in the biocompatibility of the system and in the process of molecular exchange between droplets. The recent progress in the applications of droplet-based microfluidics has been made possible by the development of new molecules and their characterizations. In this review, the role of the surfactant in droplet-based microfluidics is discussed with an emphasis on the new molecules developed specifically to overcome the limitations of ‘standard’ surfactants. Emulsion properties and interfacial rheology of surfactant-laden layers strongly determine the overall capabilities of the technology. Dynamic properties of droplets, interfaces and emulsions are therefore very important to be characterized, understood and controlled. In this respect, microfluidic systems themselves appear to be very powerful tools for the study of surfactant dynamics at the time- and length-scale relevant to the corresponding microfluidic applications. More generally, microfluidic systems are becoming a new type of experimental platform for the study of the dynamics of interfaces in complex systems.

1.The term 'surfactant' is the contraction of 'surface active agent'. It can decrease the surface tension between two phases and prevent coalescence of droplets. The author stated applications about different oils and its surfactants in tables. 

2.For droplet stability, two important aging mechanisms are marangoni effect and Ostwald ripening. The thin oil layer between droplets is the key point for droplet stability. Without surfactant, the oil will be drained and the droplets are unstable.

2-1.The marangoni effect is the mass transfer along an interface between two fluids due to a gradient of the surface tension. 

2-2.The small droplets having higher Laplace pressure than larger droplets tend to dissolve in the larger ones leading to an increase of droplet size in time and coarsening of the emulsion.

3.The non-uniform surface concentration leads to a gradient in surface tension which generates a stress opposed to the flow.

4.silicone oil has poor compatibility while hydrocarbon oil and fluorinated oil are widely used in microfluidic devices. However in hydrocarbon system, hydrophobic compounds can phase partition into the oil. The compounds encapsulated in the oil may exchanged between the droplets. Fluorinated oil  has two main advantages: first, most organic compounds are insoluble in these oils. The second advantange is biocompatibility.

\part{Biocompatible surfactants for water-in-fluorocarbon emulsions}

Drops of water-in-fluorocarbon emulsions have great potential for compartmentalizing both in vitro and in vivo biological systems; however, surfactants to stabilize such emulsions are scarce. Here we present a novel class of fluorosurfactants that we synthesize by coupling oligomeric perfluorinated polyethers (PFPE) with polyethyleneglycol (PEG).We demonstrate that these block copolymer surfactants stabilize water-in-fluorocarbon oil emulsions during all necessary steps of a drop-based experiment including drop formation, incubation, and reinjection into a second microfluidic device. Furthermore, we show that aqueous drops stabilized with these surfactants can be used for in vitro translation (IVT), as well as encapsulation and incubation of single cells. The compatability of this emulsion system with both biological systems and polydimethylsiloxane (PDMS) microfluidic devices makes these surfactants ideal for a broad range of high-throughput, drop-based applications.

1.This article mainly described EA surfactant(PFPE-PEG-PFPE ) synthesis, oil biocompatible and emulsion stability.

2.PFPE are soluble in fluorocarbon oils and large to privide stabilization pf the emulsion. However, PFPE have ionic headgroups which may interact with oppositely charged biomolecules, such as DNA, RNA and proteins.

3.PEG moieties are soluble in water and prevent adsoption of bilological compounds to interfaces.

4.The PFPE fluorocarbon chosen for the tail while PEG moieties chosen for the headgroup. Two PFPE tails having a greater interfacial anchoring strength improve emulsion stability. 

\part{1-Million droplet array with wide-field fluorescence imaging for digital PCR}

Digital droplet reactors are useful as chemical and biological containers to discretize reagents into picolitre or nanolitre volumes for analysis of single cells, organisms, or molecules. However, mostDNA based assays require processing of samples on the order of tens of microlitres and contain as few as one to as many as millions of fragments to be detected. Presented in this work is a droplet microfluidic platform and fluorescence imaging setup designed to better meet the needs of the high-throughput and high-dynamic-range by integrating multiple high-throughput droplet processing schemes on the chip. The design is capable of generating over 1-million, monodisperse, 50 picolitre droplets in 2–7 minutes that then self-assemble into high density 3-dimensional sphere packing configurations in a large viewing chamber for visualization and analysis. This device then undergoes on-chip polymerase chain reaction (PCR) amplification and fluorescence detection to digitally quantify the sample’s nucleic acid contents. Wide-field fluorescence images are captured using a low cost 21-megapixel digital camera and macro- lens with an 8–12 cm2
field-of-view at 1? to 0.85? magnification, respectively. We demonstrate both
end-point and real-time imaging ability to perform on-chip quantitative digital PCR analysis of the entire droplet array. Compared to previous work, this highly integrated design yields a 100-fold increase in the number of on-chip digitized reactors with simultaneous fluorescence imaging for digital PCR based assays.


1.The continuous phase is heavy white mineral oil with 3\% w/w Abil EM 90 and 0.1\% w/w Triton X-100.

2.The droplets are collected into a chamber full of oil. 
The thoroughput can be improved by extracting oil from the chip.

3. It provides a wide-field fluorescence imaging and on chip PCR thermocycling which can be realize in our lab.

\part{efficient extraction of oil from droplet microfluific emulsions}
Droplet microfluidic techniques can perform large numbers of single molecule and cell reactions but often require controlled, periodic flow to merge, split, and sort droplets. Here, we describe a simple method to convert aperiodic flows into periodic ones. Using an oil extraction module, we efficiently remove oil from emulsions to readjust the droplet volume fraction, velocity, and packing, producing periodic flows. The extractor acts as a universal adaptor to connect microfluidic modules that do not operate under identical flow conditions, such as droplet generators, incubators, and merger devices.

1.The oil extractor consists of main and extraction channels connnected by thin drainage channels. The connecting channels between main and extraction channels are narrow and short to extract a large fraction of oil while maintaining the droplets in the main channel.

2.Negative pressure is applied to the extractor outlet.
\end{document}
